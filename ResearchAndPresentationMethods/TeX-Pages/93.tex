\documentclass[a4paper]{book}
\usepackage{setspace}
\usepackage{scrextend}
 
\begin{document}
\setcounter{page}{93}
\markright{\hfill SEMI-STRUCTURED AND UNSTRUCTURED INTERVIEWS \hfill }

\begin{addmargin}{2em}
\vspace{5mm}

\begin{spacing}{1.00}
\noindent {\small \textsf{$\bullet$ Respond to this email indicating that you have read and understood the purpose of the study and that you agree to participate.}}
\end{spacing}


\vspace{5mm}
\begin{spacing}{1.00}
\noindent {\small \textsf{$\bullet$ Participate in an online interview that will require approximately 10-15 minutes of your time for about 14 consecutive days (May 7-21).}}
\end{spacing}


\vspace{5mm}
\begin{spacing}{1.00}
\noindent {\small \textsf{All Internet communication has an element of risk-the possibility that others will intentionally or unintentionally access the data. However, I will take steps to provide confidentiality through placing the data on a secured server, and all names and identifying characteristics will be removed from corresponding reports and publications. No deception will be used, and you are free to withdraw from the interview at any point or refuse to answer any questions. The interview data will be destroyed after a period of five years unless permission has been granted for further use. A summary of the research will be posted on a Web site at the completion of the study, and I will email the URL to you. If you would like further information, you may contact me at [email and/or phone] or my supervisor [administrator, chair, or dean] at [email and/or phone].}}
\end{spacing}



\vspace{5mm}
\begin{spacing}{1.00}
\noindent {\small \textsf{If you agree to participate in this study, please reply to this email confirming that you have read and understood the purpose of this study, understand your rights as a participant, and agree to participate. Thank you.}}
\end{spacing}



\begin{flushleft}
{\small \textsf{Regards,}}
\end{flushleft}


\begin{flushleft}
\begin{spacing}{2.00}
{\small \textsf{[e-researcher]}}
		
{\small \textsf{Principle Researcher}}
\end{spacing}
\end{flushleft}
\end{addmargin}

The email should be written in a conversational style in an effort to establish rapport and trust. As mentioned, it is important that the e-researcher take steps to put the respondent at ease and enable him or her to feel comfortable, setting the stage for a positive and responsive interview process.

The next correspondence should begin the interview process. The first way to gather meaningful data, prior to asking questions, is to establish an informal and friendly conversational tone (Kvale, 1996). Some people find that setting a friendly tone in Net-based interviews is not as easy (nor as familiar) as in face-to-face interviews. However, there are ways to facilitate this process, including the appropriate use of humor, self-disclosure, or narrative (Baym, 1995). We suggest that the first email posting should set the tone for the conversation between the e-researcher and the participant. We have also found that it can be useful to share our own uncertainty and nervousness in conducting Net-based interviews with participants.

The following is an example of a welcome message that sets a friendly tone. When the message is kept to one screen of text and is informal and friendly, the likelihood


\end{document}
