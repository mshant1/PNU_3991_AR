\documentclass[a4paper]{book}
\usepackage{setspace}
\usepackage{scrextend}

 
\begin{document}
\setcounter{page}{91}
\markright{\hfill SEMI-STRUCTURED AND UNSTRUCTURED INTERVIEWS \hfill }

 Good qualitative research studies not only keep accurate records of the methods, procedures, and the evolving analysis, but are also self-reflective, resulting in a balance between reflective and descriptive material (Bogdan \& Biklen, 1992). Fontana and Frey (1994) describe how reflective activity benefits the e-researcher: {``}As we treat the other as a human being, we can no longer remain objective, faceless interviewers, but become human beings and must disclose ourselves, learning about ourselves as we try to learn about the other'' (p. 374). Becoming aware of and understanding our biases and assumptions through a reflective journal are essential activities that help us articulate not only what we do during interviews, but why. Knowing why is essential when analyzing the data, as our assumptions and biases influence what data we select and report as well as what insights we generate, and how.




\begin{flushleft}
\vspace{5mm}
\hspace{-2cm}
\textbf{{\Large INITIATING THE PROCESS}}
\vspace{5mm}
\end{flushleft}


Inviting someone to participate in an interview is extending an invitation to participate in a conversation and build a relationship (Weber, 1986). The e-researcher normally begins the interview process with a letter of invitation. We suggest a paper-based, mailed letter on the e-researcher's affiliated institution letterhead. Because Net-based interviews are conducted in a virtual environment, the e-researcher should provide the participants with evidence that he or she is conducting valid research affiliated with a credible institution. Using institutional letterhead is one way to provide such evidence. The letter should reflect the nature of the research process, describe the purpose of the study, and include details with respect to what is being requested of the participant in terms of time and Internet resources. The language should be appropriate to the participants, not the researcher, and thus be simple, direct, and straightforward. The following is a sample letter of invitation.

\begin{addmargin}{2em}

\begin{flushleft}
\vspace{5mm}
\begin{center}
\textbf{{\small \textsf{Sample Letter of Invitation}}}
\end{center}
\end{flushleft}



\begin{flushleft}
{\small \textsf{[date]}}
\end{flushleft}


\begin{flushleft}
{\small \textsf{Dear \_\_\_\_\_\_\_}}
\end{flushleft}



\begin{spacing}{1.00}
\noindent {\small \textsf{I would like to invite you to participate in an }}\textbf{{\small \textsf{online }}}{\small \textsf{interview. The objective of this study is to explore the transformation that e-learning is having on higher education communities in North America. In order to fully understand the perspectives, needs, and concerns of educators like yourself, it is critical that your comments be heard and understood by researchers and decision makers.}}

\vspace{5mm}
\noindent {\small \textsf{I anticipate the interview will require approximately 10-15 minutes of your time for about 14 consecutive days (May 7-21), to respond to my questions. It is not necessary that you respond every consecutive day, but you should be prepared to respond at least every other day. You will need access to the Internet and an email}}
\end{spacing}

\end{addmargin}

\end{document}
